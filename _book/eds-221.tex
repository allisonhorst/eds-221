% Options for packages loaded elsewhere
\PassOptionsToPackage{unicode}{hyperref}
\PassOptionsToPackage{hyphens}{url}
%
\documentclass[
]{book}
\usepackage{lmodern}
\usepackage{amsmath}
\usepackage{ifxetex,ifluatex}
\ifnum 0\ifxetex 1\fi\ifluatex 1\fi=0 % if pdftex
  \usepackage[T1]{fontenc}
  \usepackage[utf8]{inputenc}
  \usepackage{textcomp} % provide euro and other symbols
  \usepackage{amssymb}
\else % if luatex or xetex
  \usepackage{unicode-math}
  \defaultfontfeatures{Scale=MatchLowercase}
  \defaultfontfeatures[\rmfamily]{Ligatures=TeX,Scale=1}
\fi
% Use upquote if available, for straight quotes in verbatim environments
\IfFileExists{upquote.sty}{\usepackage{upquote}}{}
\IfFileExists{microtype.sty}{% use microtype if available
  \usepackage[]{microtype}
  \UseMicrotypeSet[protrusion]{basicmath} % disable protrusion for tt fonts
}{}
\makeatletter
\@ifundefined{KOMAClassName}{% if non-KOMA class
  \IfFileExists{parskip.sty}{%
    \usepackage{parskip}
  }{% else
    \setlength{\parindent}{0pt}
    \setlength{\parskip}{6pt plus 2pt minus 1pt}}
}{% if KOMA class
  \KOMAoptions{parskip=half}}
\makeatother
\usepackage{xcolor}
\IfFileExists{xurl.sty}{\usepackage{xurl}}{} % add URL line breaks if available
\IfFileExists{bookmark.sty}{\usepackage{bookmark}}{\usepackage{hyperref}}
\hypersetup{
  pdftitle={EDS 221: Scientific Programming Essentials},
  pdfauthor={Allison Horst},
  hidelinks,
  pdfcreator={LaTeX via pandoc}}
\urlstyle{same} % disable monospaced font for URLs
\usepackage{longtable,booktabs}
\usepackage{calc} % for calculating minipage widths
% Correct order of tables after \paragraph or \subparagraph
\usepackage{etoolbox}
\makeatletter
\patchcmd\longtable{\par}{\if@noskipsec\mbox{}\fi\par}{}{}
\makeatother
% Allow footnotes in longtable head/foot
\IfFileExists{footnotehyper.sty}{\usepackage{footnotehyper}}{\usepackage{footnote}}
\makesavenoteenv{longtable}
\usepackage{graphicx}
\makeatletter
\def\maxwidth{\ifdim\Gin@nat@width>\linewidth\linewidth\else\Gin@nat@width\fi}
\def\maxheight{\ifdim\Gin@nat@height>\textheight\textheight\else\Gin@nat@height\fi}
\makeatother
% Scale images if necessary, so that they will not overflow the page
% margins by default, and it is still possible to overwrite the defaults
% using explicit options in \includegraphics[width, height, ...]{}
\setkeys{Gin}{width=\maxwidth,height=\maxheight,keepaspectratio}
% Set default figure placement to htbp
\makeatletter
\def\fps@figure{htbp}
\makeatother
\setlength{\emergencystretch}{3em} % prevent overfull lines
\providecommand{\tightlist}{%
  \setlength{\itemsep}{0pt}\setlength{\parskip}{0pt}}
\setcounter{secnumdepth}{5}
\usepackage{booktabs}
\ifluatex
  \usepackage{selnolig}  % disable illegal ligatures
\fi
\usepackage[]{natbib}
\bibliographystyle{apalike}

\title{EDS 221: Scientific Programming Essentials}
\author{Allison Horst}
\date{2021-03-17}

\begin{document}
\maketitle

{
\setcounter{tocdepth}{1}
\tableofcontents
}
\hypertarget{scientific-programming-essentials-for-environmental-data-science}{%
\chapter{Scientific programming essentials for environmental data science}\label{scientific-programming-essentials-for-environmental-data-science}}

As nicely summarized in the title of a \href{https://www.nceas.ucsb.edu/news/next-generation-environmental-scientists-are-data-scientists}{2018 NCEAS post}: \emph{``The next generation of environmental scientists are data scientists''}.

Over the next year in MEDS you'll build skills to responsibly apply advanced methods in environmental modeling, spatial data analysis, and machine learning to investigate and solve environmental problems.

To get there, however, you'll need a strong foundation in programming basics like: understanding types and structures of data, basic data wrangling and visualization, algorithm development with functions, loops, and conditionals, and how to troubleshoot. While working in the weeds of programming, we'll also learn and reinforce transferable habits for reproducible workflows, robust file paths, version control, data organization, and more.

Upon these building blocks established in EDS 221, you'll be able to incrementally grow your advanced environmental data science toolkit to enter the workplace at the leading edge of quantitative methods in the field.

\hypertarget{setup}{%
\chapter{Setup}\label{setup}}

Intro to programming and the tools we're using.

\hypertarget{types}{%
\chapter{Data types and structures}\label{types}}

Data structures info\ldots{}

\hypertarget{methods}{%
\chapter{Methods}\label{methods}}

We describe our methods in this chapter.

\hypertarget{iteration}{%
\chapter{Iteration}\label{iteration}}

Iteration

\hypertarget{conditionals}{%
\chapter{Conditionals}\label{conditionals}}

\hypertarget{logicals}{%
\chapter{Logicals}\label{logicals}}

\hypertarget{functions}{%
\chapter{Functions}\label{functions}}

\hypertarget{tidydata}{%
\chapter{Tidy data}\label{tidydata}}

\hypertarget{tidyverse}{%
\chapter{Data wrangling \& viz in the tidyverse}\label{tidyverse}}

\hypertarget{troubleshooting}{%
\chapter{Troubleshooting}\label{troubleshooting}}

  \bibliography{book.bib,packages.bib}

\end{document}
